
%% Basierend auf einer TeXnicCenter-Vorlage von Mark Müller
%%%%%%%%%%%%%%%%%%%%%%%%%%%%%%%%%%%%%%%%%%%%%%%%%%%%%%%%%%%%%%%%%%%%%%%

% Wählen Sie die Optionen aus, indem Sie % vor der Option entfernen  
% Dokumentation des KOMA-Script-Packets: scrguide

%%%%%%%%%%%%%%%%%%%%%%%%%%%%%%%%%%%%%%%%%%%%%%%%%%%%%%%%%%%%%%%%%%%%%%%
%% Optionen zum Layout des Artikels                                  %%
%%%%%%%%%%%%%%%%%%%%%%%%%%%%%%%%%%%%%%%%%%%%%%%%%%%%%%%%%%%%%%%%%%%%%%%
\documentclass[%
%a5paper,							% alle weiteren Papierformat einstellbar
%landscape,						% Querformat
12pt,								% Schriftgröße (12pt, 11pt (Standard))
%BCOR1cm,							% Bindekorrektur, bspw. 1 cm
%DIVcalc,							% führt die Satzspiegelberechnung neu aus
%											  s. scrguide 2.4
%twoside,							% Doppelseiten
%twocolumn,						% zweispaltiger Satz
%halfparskip*,				% Absatzformatierung s. scrguide 3.1
%headsepline,					% Trennline zum Seitenkopf	
%footsepline,					% Trennline zum Seitenfuß
titlepage,						% Titelei auf eigener Seite
%normalheadings,			% Überschriften etwas kleiner (smallheadings)
%idxtotoc,						% Index im Inhaltsverzeichnis
%liststotoc,					% Abb.- und Tab.verzeichnis im Inhalt
%bibtotoc,						% Literaturverzeichnis im Inhalt
%abstracton,					% Überschrift über der Zusammenfassung an	
%leqno,   						% Nummerierung von Gleichungen links
%fleqn,								% Ausgabe von Gleichungen linksbündig
%draft								% überlangen Zeilen in Ausgabe gekennzeichnet
bibliography=totoc
]
{scrartcl}

%\pagestyle{empty}		% keine Kopf und Fußzeile (k. Seitenzahl)
%\pagestyle{headings}	% lebender Kolumnentitel  


%% Deutsche Anpassungen %%%%%%%%%%%%%%%%%%%%%%%%%%%%%%%%%%%%%
\usepackage[ngerman]{babel}
\usepackage[T1]{fontenc}
\usepackage[ansinew]{inputenc}
\usepackage{graphicx}

\usepackage{lmodern} %Type1-Schriftart für nicht-englische Texte
\usepackage{fancyhdr}

\usepackage{natbib}
%\usepackage{cite}
%\renewcommand{\citeleft}{}
%\renewcommand{\citeright}{}

%% Packages für Grafiken & Abbildungen %%%%%%%%%%%%%%%%%%%%%%
\usepackage{graphicx} %%Zum Laden von Grafiken
%\usepackage{subfig} %%Teilabbildungen in einer Abbildung
%\usepackage{pst-all} %%PSTricks - nicht verwendbar mit pdfLaTeX

%% Beachten Sie:
%% Die Einbindung einer Grafik erfolgt mit \includegraphics{Dateiname}
%% bzw. über den Dialog im Einfügen-Menü.
%% 
%% Im Modus "LaTeX => PDF" können Sie u.a. folgende Grafikformate verwenden:
%%   .jpg  .png  .pdf  .mps
%% 
%% In den Modi "LaTeX => DVI", "LaTeX => PS" und "LaTeX => PS => PDF"
%% können Sie u.a. folgende Grafikformate verwenden:
%%   .eps  .ps  .bmp  .pict  .pntg


%% Bibliographiestil %%%%%%%%%%%%%%%%%%%%%%%%%%%%%%%%%%%%%%%%%%%%%%%%%%
%\usepackage{natbib}

\begin{document}
\parskip=1em
\parindent=0cm
\pagestyle{empty} %%Keine Kopf-/Fusszeilen auf den ersten Seiten.


%%%%%%%%%%%%%%%%%%%%%%%%%%%%%%%%%%%%%%%%%%%%%%%%%%%%%%%%%%%%%%%%%%%%%%%
%% Ihr Artikel                                                       %%
%%%%%%%%%%%%%%%%%%%%%%%%%%%%%%%%%%%%%%%%%%%%%%%%%%%%%%%%%%%%%%%%%%%%%%%

%% eigene Titelseitengestaltung %%%%%%%%%%%%%%%%%%%%%%%%%%%%%%%%%%%%%%%    
%\begin{titlepage}
%Einsetzen der TXC Vorlage "Deckblatt" möglich
%description: Deckblatt in Deutsch
%% Basierend auf einer TeXnicCenter-Vorlage von Tino Weinkauf.
%%%%%%%%%%%%%%%%%%%%%%%%%%%%%%%%%%%%%%%%%%%%%%%%%%%%%%%%%%%%%%

%%%%%%%%%%%%%%%%%%%%%%%%%%%%%%%%%%%%%%%%%%%%%%%%%%%%%%%%%%%%%
%% Deckblatt
%%%%%%%%%%%%%%%%%%%%%%%%%%%%%%%%%%%%%%%%%%%%%%%%%%%%%%%%%%%%%
%%
%% ACHTUNG: Sie ben�tigen ein Hauptdokument, um diese Datei
%%          benutzen zu k�nnen. Verwenden Sie im Hauptdokument
%%          den Befehl "\input{dateiname}", um diese
%%          Datei einzubinden.
%%

\begin{titlepage}
\vspace{5cm}
Stefan Waidele\\
Immatrikulationsnummer 1028171\\
Ensisheimer Stra�e 2\\
79395 Neuenburg am Rhein\\
Stefan.Waidele@AKAD.de

\vfill

\begin{tabbing}
Modul ANS09 --- \=Betriebswirtschaftliche Standardsoftware\\
                \>und Enterprise Application Integration (EAI)\\ \\
                \>Assignment  
\end{tabbing}
\LARGE
\textsc{Integrationsanforderungen von \\Informationssystemen am Beispiel\\eines Hotel-- und Restaurantbetriebs}\\

\vfill

\normalsize
Betreuer: Prof. Ulrich Gr�ff

\today %%Datum der Abgabe - am besten selbst reinschreiben.

\vfill

\includegraphics[width=3cm]{AKAD-Logo}

AKAD Hochschule Stuttgart

\end{titlepage}
%\end{titlepage}

%% Erzeugung von Verzeichnissen %%%%%%%%%%%%%%%%%%%%%%%%%%%%%%%%%%%%%%%
\tableofcontents			% Inhaltsverzeichnis
%\listoftables				% Tabellenverzeichnis
%\listoffigures				% Abbildungsverzeichnis
\newpage


%% Der Text %%%%%%%%%%%%%%%%%%%%%%%%%%%%%%%%%%%%%%%%%%%%%%%%%%%%%%%%%%%
\pagestyle{headings}
\section{Einleitung}
\subsection{Ziel der Untersuchung}

Ziel dieser Arbeit ist die Analyse und Bewertung der Integrationsanforderungen des Hotelinformationssystems anhand definierter Anforderungsstufen und Integrationsgegenst�nden.

\subsection{Vorgehensweise}

Eingangs des 2. Kapitels erfolgt eine kurze Beschreibung der f�r diese Arbeit relevanten Eigenschaften des Unternehmens. Danach werden die f�r das Verst�ndnis der vorliegenden Arbeit wichtigen Begriffe erl�utert. Mit der Herausarbeitung von Anforderungsstufen und der Integrationsgegenst�nde schlie�t der theoretische Teil ab.

Zu Beginn des praktischen Teils werden die zwei zu betrachtenden Gesch�ftsprozesse beschrieben. Danach folgen die wichtigsten Schnittstellen bzw. Medienbr�che. Den Abschluss dieses Kapitels stellt eine Bewertung  der Schnittstellen anhand der im 2. Kapitel beschriebenen Kriterien. Zum Abschluss der Arbeit werden die Ergebnisse zusammengefasst.

\subsection{Abgrenzung}

Nicht alle unterst�tzte Gesch�ftsprozesse des Hotelinformationssystems flie�en in die Analyse ein. In dieser Arbeit soll das Hauptaugenmerk auf zwei Kernprozesse im Unternehmen gelegt werden, anhand derer die Integrationsanforderungen in einem kleinen Betrieb deutlich werden. Diese werden auch nicht formell modelliert werden. Das Augenmerk der Betrachtungen liegt an den zu erfassenden Daten, deren Weitergabe und die daf�r notwendigen Schnittstellen bzw. die bestehenden Medienbr�chen. 

Aufgrund des sehr geringen Integrationsgrad, welcher f�r Gastronomische Betrieber dieser Gr��enordnung durchaus �blich ist, liegt das Hauptaugenmerk dieser Arbeit auf der Datenintegration. Auch wird keine Wirtschaftlichkeitsberechnung durchgef�hrt. 

Auch wird der fachliche und der Marketingaspekt in dieser Arbeit nicht betrachtet, da diese nur geringen Einfluss auf die Integration haben. Die ebenfalls im Betrieb vorhandenen Abteilungen "`Metzgerei"' und "`Tagungen"' bleiben g�nzlich unbeachtet.

Diese Arbeit besch�ftigt sich mit den folgenden Situationen der betrieblichen Leistungserbringung:

\begin{itemize}
	\item \textbf{Verkauf von Speisen \& Getr�nken im Restaurant:}\\
	Dieser Prozess beschreibt einen gro�en Ausschnitt der kundenzugewandte Seiten der Leistungserstellung im Restaurant. F�r diese Arbeit relevent sind die M�glichkeiten der abteilungs�bergreifenden Integration sowie die Schnittstellen f�r die Daten�bergabe an die Hotelabteilung (horizontale Integration) bzw. an das Management (vertikale Integration) betrachtet.
	\item \textbf{Hotelreservierung}\\
Hier stehen weniger die Erfassung und Verwaltung der G�stedaten und Reservierungen im Blickpunkt, sondern der Umgang mit den Informationen �ber noch zu vermietende Zimmer und die anzusetzenden Preise.
\end{itemize}


\clearpage
\section{Grundlagen}

\subsection{Vorstellung des Unternehmens}

Der Autor untersucht in dieser Arbeit das im eigenen Unternehmen eingesetzte Hotel-- und Restaurantinformationssystem. 
Das Unternehmen ist ein �ber mehere Generationen in Familienbesitz gef�hrtes Einzelunternehmen, 
bestehend aus den Abteilungen Hotel, Restaurant, Metzgerei und Tagungsbereich. Mit 90 Betten und 100 Sitzpl�tzen ist
der Betrieb als "`kleines Hotel"' einzustufen. Mit ca. 30 Mitarbeitern ist das Unternehmen allerdings auch kein reiner Familienbetrieb mehr.

\subsection{Wichtige Fachtermini und Abk�rzungen}

\begin{itemize}
	\item \textbf{POS:} Point of Sale --- Verkaufsstelle. In der Gastronomie wird mit dem Begriff POS--System i.d.R. die Registrierkasse im Restaurant bezeichnet, in der die Bestellung erfasst, gespeichert und auf den verschiedenen Bondrucker ausgegeben wird.
	\item \textbf{Front--Office:} Im Hotel bezeichnet man mit Front--Office den Bereich der Rezeption, an dem der Mitarbeiter in direktem Kontakt zum Kunden steht. F�r diese Arbeit schlie�t dies nicht nur den pers�nlichen Kontakt vor Ort ein, sondern auch die Bereiche der Reservierung und sonstige Korrespondenz, welche in gro�en Hotels auch im \textbf{Back--Office} (also dem B�ro ohne direkten G�stekontakt) oder auch der Abteilung Reservierung zugeordnet sein k�nnen.
	
Trotz deutlichen Unterschieden in der Benutzung und in der Wahrnehmung k�nnen die Begriffe POS-System und Front--Office System somit abteilungsspezifische Bezeichnungen f�r �quivalente Systeme gesehen werden, bei denen der Kundenauftrag f�r die weitere Verarbeitung per EDV erfasst wird. 
	
	\item \textbf{Hotel-- und Restaurantinformationssystem (HRIS):} Betriebliches Informationssystem das die Anforderungen eines Hotels mit Restaurantbetriebs erf�llt.
	\item \textbf{Service Oriented Architecture / SOA:} TODO: Definition!!!
\end{itemize}

\subsection{Anforderungsstufen}

Im Rahmen dieser Arbeit werden die Integrationsanforderungen des Hotel-- und Restaurantinformationssystems auf verschiedenen Stufen betrachtet werden:

\begin{itemize}
	\item Minimalanforderung: Diese beschreiben den Mindestgrad an Integration die f�r einen Betrieb dieser Gr��enordnung wirtschaftlich Sinnvoll ist. 
	\item Momentan umgesetzte Ingegrationsanforderungen: Dies beschreibt den Ist--Zustand im untersuchten Betrieb.
	\item Optimalanforderungen: Hier werden die M�glichkeiten eines hochintegrierten HRIS aufgezeigt.
\end{itemize}

\subsection{Integration}

Integration bedeutet die Verbindung einer Vielheit zu einer Einheit.\footnote{vgl. \cite{dudenint}}

In der Informatik ist demnach Integration als die Verbindung von mehreren getrennten Anwendungssystemen zu einem integrierten Anwendungssystem, welches Aufgaben aus verschiedenen Funktionsbereichen und die verschiedenen Bereiche intern zu einem Gesamtsystem verkn�pft. Hierbei werden Daten m�glichst fr�h erfasst und dann systemintern verarbeitet, gespeichert und weitergeleitet.\footnote{vgl. \cite{staud}, S.30}

Die Integration von Informationssystemen kann auf verschiedenen Arten geschehen. Hierbei sind die unterschiedlichen  Integrationsgegenst�nde jeweils mit den Integrationsrichtungen sowie die der Integrationsreichweite zu betrachten.

\subsection{Integrationsgegenst�nde}

vgl. \cite{staud} Seite 36f

\begin{itemize}
	\item \textbf{Datenintegration:} Hierbei werden die Daten meherer Betriebsbereiche zusammengef�hrt, um dann mit verschiedenen Programmen auf die gemeinsame Datenbasis zugreifen zu k�nnen. \footnote{vgl. \cite{gablerdint}}
	\item \textbf{Funktionsintegration:} Hierbei werden meherere betriebliche Teilfunktionen zusammengef�hrt. So k�nnen vorher getrennte Aufgaben anschlie�end an einem Arbeitsplatz ausgef�hrt werden. \footnote{\cite{gablerfint}}
	\item \textbf{Prozessintegration:} Informationstechnische Verbindung zwischen einzelnen Vorg�ngen, z.B. Auftragserfassung und Materialbeschaffung.
	\item \textbf{Methodenintegration:} Die einzelnen Funktionen werden so gesteltet, dass die Ergebnisse von Funktionen direkt als Eingabe der n�chsten Funktionen genutzt werden k�nnen.
	\item \textbf{Programmintegration:} Realisierung der aufeinander abgestimmten Softwaremodule. Aufgabe des Software--Engineering.
\end{itemize}

\subsection{Integrationsrichtungen}

Bei der horizontalen Integration werden (Teil)prozesse einer Managementebene aus verschiedenen Funktionsbereichen des Unternehmens miteinander verkn�pft. Bei der Vertikalen Integration werden die Systeme �ber die grenzen der Managementebenen hinaus verkn�pft, so dass z.B. das F�hrungsinformationssystem direkt auf die Daten der Produktion zugreigen kann.\footnote{vgl. \cite{staud} Seite 37}

\subsection{Integrationsreichweite}

Hierunter versteht man die l�nge der Integrierten Prozessketten. Dies beginnt von abteilungsinternen Integrationsschritten �ber die Integration von Vorg�ngen �ber mehrere Abteilungen hinweg, bis hin zu Unternehmens�bergreifenden Projekten zum Suppy Chain Management, in denen sowohl Zulieferer als auch Kunden beteiligt sein k�nnen.
\clearpage
\section{Praktischer Teil}

TODO: Beschreibung der zu betrachtenden Gesch�ftsprozessen und der genutzten Informationssystemen.

In dieser Arbeit soll das Hauptaugenmerk auf zwei Kernprozesse im Unternehmen gelegt werden, anhand derer die Integrationsanforderungen in einem kleinen Betrieb deutlich werden. Dies geschieht ohne genaue Modellierung des fachlichen oder des Marketingaspekts, da diese nur bedingt Einfluss auf die Integration haben. Das Augenmerk der Betrachtungen liegt an den zu erfassenden Daten, deren Weitergabe und die daf�r notwendigen Schnittstellen. 

\begin{itemize}
	\item \textbf{Verkauf von Speisen \& Getr�nken im Restaurant:}\\
	Dieser Prozess beschreibt einen gro�en Ausschnitt der kundenzugewandte Seiten der Leistungserstellung im Restaurant. F�r diese Arbeit relevent sind die M�glichkeiten der abteilungs�bergreifenden Integration sowie die Schnittstellen f�r die Daten�bergabe an die Hotelabteilung (horizontale Integration) bzw. an das Management (vertikale Integration) betrachtet.
	\item \textbf{Hotelreservierung}\\
Hier stehen weniger die Erfassung und Verwaltung der G�stedaten und Reservierungen im Blickpunkt, sondern der Umgang mit den Informationen �ber noch zu vermietende Zimmer und die anzusetzenden Preise.
\end{itemize}

\subsection{Verkauf von Speisen \& Getr�nken im Restaurant}

TODO: Hier fehlt die eEPK / bzw. die eEPKs / oder doch nicht?

Das im Einsatz befindliche POS-System "`Casio 6000"' besteht aus einer propriet�ren Restaurantkasse, in der sowohl die Stammdaten als auch die im laufenden Betrieb anfallenden Bewegungsdaten gespeichert werden. Auch das Reporting und eine beschr�nkte Archivierungsfunktion wird erf�llt.

Die Stammdaten basieren auf einer Artikelliste in der neben der Artikelnummer auch eine Beschreibung sowie der Preis gespeichert ist. Aber auch steuerlich relevante Daten wie der Mehrwertsteuersatz sowie Parameter des Customizing (z.B. auf welchem Bondrucker die Bestellung ausgegeben wird) sind hier hinterlegt. 

Nach der Bestellannahme bucht die Servicekraft die gew�nschten Artikel (Auswahl per Men� oder Artikelnummer) auf den Tisch der G�ste. Das System speichert automatisch eine eindeutige Bonnummer, den zum Erfassungszeitpunkt g�ltigen Preis und die Personalnummer der Servicekraft. Diese Daten werden auch auf den ausgedruckten Bons sowie dem intern gespeichertem Journalspeicher dokumentiert.

\subsubsection{Schnittstelle zum Management --- Medienbruch}

Des Weiteren werden die angelegten Berichte weitergef�hrt. Dies ist in der Regel der Umsatzbericht pro Servicekraft und der Warengruppenbericht. Weitere Auswertungen z.B. nach Zeitintervallen k�nnen konfiguriert werden. 

Diese Berichte sind die Schnittstelle zur Buchhaltung und zum Management. Zur weiteren Verarbeitung m�ssen die ausgedruckten Daten in die entsprechenden Systeme eingegeben werden! 

Jedoch ist zu beachten, dass die Berichtsdaten zwar laufend aktualisiert werden, es jedoch notwendig ist, die Berichte manuell auszudrucken. Hierbei sind zwei Modi m�glich: Der X--Abschlag, der die aufgelaufenen Daten auswertet und ausgibt, sowie der Z--Abschlag, der die Register nach dem Ausdruck auf Null setzt. Beiden Modi ist gemeinsam, dass die Auswertung der Daten nur zum Zeitpunkt des Ausdrucks ausgewertet werden. 

So ist ein Tagesbericht tats�chlich nur tats�chlich zwischen dem Ende der letzen Schicht und vor dem Beginn der ersten Schicht des Folgetages m�glich. Der Monatsbericht muss nach dem Tagesabschluss des letzten Tags des Monats, vor dem ersten Gesch�ftsvorgang des Folgemonats durchzuf�hren. Wir dies nicht beachtet, so wird die Qualit�t der Daten deutlich verschlechtert.

Dieser Medienbruch wird auch dadurch nicht behoben, dass im Back--Office eine Software (Casio Easy Reporting) im Einsatz ist, die auf die entsprechenden Daten der Kasse zugreifen kann. Dieser Zugriff geschieht n�mlich auch nur manuell und in unstrukturierter Form, so dass auch hier umfangreiche Importma�nahmen notwendig sind und auch weiterhin die Notwendigkeit besteht, die Auswertungen zu den entsprechenden Zeitpunkten p�nktlich durchzuf�hren.

\subsubsection{Schnittstelle zur Hotelsoftware --- Medienbruch}


\subsubsection{Hotelreservierung}

\subsection{Die wichtigsten Schnittstellen}

Herausarbeitung der wichtigsten Schnittstellen zwischen den Systemen und des Nutzens, den das Unternehmen durch Integration erh�lt -> Integrationsanforderungen. 

\subsection{Bewertung der Schnittstellen}

Bewertung der Schnittstellen anhand der Anforderungsstufen und Integrationsgegenst�nden.


%\clearpage
\section{Fazit}

Zusammenfassend l�sst sich sagen, dass es sich bei dem im Restaurant eingesetzte System um eine Insell�sung handelt. Lediglich in Einzelaspekten gibt es eine automatische Daten�bergabe.

Im Hotelbereich sind dagegen schon L�sungen auf einer recht hohen Integrationsstufe im Einsatz. Lediglich bei den Schnittstellen zum Restaurant und zur abteilungs�bergreifenden Buchhaltung besteht noch Integrationspotential. 

Durch jeden weiteren Schritt zu einem h�heren Integrationsniveau werden die in dieser Arbeit beschriebenen M�ngel des momentanen Systems durch die Vorteile integrierter Systeme ersetzt. Die Prozesse werden mit weniger Benutzereingriffen, schneller und mit geringerer Fehleranf�lligkeit ausgef�hrt. 

Somit l�sst sich sagen, dass die generellen Anforderungen an die Integration auch bei kleinen gastronomischen Betrieben vorhanden sind. Jedoch ist die Notwendigkeit, die Dringlichkeit und auch die Rentabilit�t einer Umsetzung deutlich geringer als in gro�en Betrieben.

Im allgemeinen ist auch f�r kleine und mittlere Betriebe ein hochintegriertes Hotel-- und Restaurant Informationssystem anzustreben.

\section{Ausblick}

In weiteren Untersuchungen sollte der finanzielle Aspekt hochintegrierter L�sungen mit in Betracht gezogen werden, da sich hohe Investitionen in die EDV speziell bei kleinen Betrieben evt. nicht amortisieren. 

Ebenfalls sollten Komponenten anderer Softwareh�user als Alternativen zu den bislang Eingesetzten auf deren Integrationsniveau untersucht werden. Hierzu z�hlt auch die Untersuchung der in gro�en Hotels eingesetzten Systeme auf deren Eignung zum Einsatz in kleinen H�usern. Eventuell ist auch schon durch den Tausch einzelner Programme ein System realisierbar, dass deutlich besser zusammenarbeitet als das Momentane.

Betriebs�bergreifend ist die n�here Untersuchung der Channel--Manager interessant. In wie weit nehmen diese die Middleware--Funktion war und wie bedeutsam ist ihre Stellung im Gesamtmarkt. Ist die Hub and Spokes Struktur so vorteilhaft, dass gro�e Channel--Manager ihre zentrale Stellung auch in Marktmacht umsetzen k�nnen? Oder bewirkt eine weitere Standardisierung der Kommunikatinsprotokolle dass die eingenommene �bersetzungs-- und Verteilerfunktion bald �berfl�ssig wird? 


%% Bibliographie unter Verwendung von dinnat %%%%%%%%%%%%%%%%%%%%%%%%%%
%\setbibpreamble{Präambel}		% Text vor dem Verzeichnis
\bibliographystyle{natdin}
%\bibliographystyle{plaindin}
\bibliography{Literatur}	% Sie benötigen eine *.bib-Datei

\end{document}
