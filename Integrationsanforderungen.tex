
%% Basierend auf einer TeXnicCenter-Vorlage von Mark Müller
%%%%%%%%%%%%%%%%%%%%%%%%%%%%%%%%%%%%%%%%%%%%%%%%%%%%%%%%%%%%%%%%%%%%%%%

% Wählen Sie die Optionen aus, indem Sie % vor der Option entfernen  
% Dokumentation des KOMA-Script-Packets: scrguide

%%%%%%%%%%%%%%%%%%%%%%%%%%%%%%%%%%%%%%%%%%%%%%%%%%%%%%%%%%%%%%%%%%%%%%%
%% Optionen zum Layout des Artikels                                  %%
%%%%%%%%%%%%%%%%%%%%%%%%%%%%%%%%%%%%%%%%%%%%%%%%%%%%%%%%%%%%%%%%%%%%%%%
\documentclass[%
%a5paper,							% alle weiteren Papierformat einstellbar
%landscape,						% Querformat
12pt,								% Schriftgröße (12pt, 11pt (Standard))
%BCOR1cm,							% Bindekorrektur, bspw. 1 cm
%DIVcalc,							% führt die Satzspiegelberechnung neu aus
%											  s. scrguide 2.4
%twoside,							% Doppelseiten
%twocolumn,						% zweispaltiger Satz
%halfparskip*,				% Absatzformatierung s. scrguide 3.1
%headsepline,					% Trennline zum Seitenkopf	
%footsepline,					% Trennline zum Seitenfuß
titlepage,						% Titelei auf eigener Seite
%normalheadings,			% Überschriften etwas kleiner (smallheadings)
%idxtotoc,						% Index im Inhaltsverzeichnis
%liststotoc,					% Abb.- und Tab.verzeichnis im Inhalt
%bibtotoc,						% Literaturverzeichnis im Inhalt
%abstracton,					% Überschrift über der Zusammenfassung an	
%leqno,   						% Nummerierung von Gleichungen links
%fleqn,								% Ausgabe von Gleichungen linksbündig
%draft								% überlangen Zeilen in Ausgabe gekennzeichnet
bibliography=totoc
]
{scrartcl}

%\pagestyle{empty}		% keine Kopf und Fußzeile (k. Seitenzahl)
%\pagestyle{headings}	% lebender Kolumnentitel  


%% Deutsche Anpassungen %%%%%%%%%%%%%%%%%%%%%%%%%%%%%%%%%%%%%
\usepackage[ngerman]{babel}
\usepackage[T1]{fontenc}
\usepackage[ansinew]{inputenc}
\usepackage{graphicx}

\usepackage{lmodern} %Type1-Schriftart für nicht-englische Texte
\usepackage{fancyhdr}

\usepackage{natbib}
%\usepackage{cite}
%\renewcommand{\citeleft}{}
%\renewcommand{\citeright}{}

%% Packages für Grafiken & Abbildungen %%%%%%%%%%%%%%%%%%%%%%
\usepackage{graphicx} %%Zum Laden von Grafiken
%\usepackage{subfig} %%Teilabbildungen in einer Abbildung
%\usepackage{pst-all} %%PSTricks - nicht verwendbar mit pdfLaTeX

%% Beachten Sie:
%% Die Einbindung einer Grafik erfolgt mit \includegraphics{Dateiname}
%% bzw. über den Dialog im Einfügen-Menü.
%% 
%% Im Modus "LaTeX => PDF" können Sie u.a. folgende Grafikformate verwenden:
%%   .jpg  .png  .pdf  .mps
%% 
%% In den Modi "LaTeX => DVI", "LaTeX => PS" und "LaTeX => PS => PDF"
%% können Sie u.a. folgende Grafikformate verwenden:
%%   .eps  .ps  .bmp  .pict  .pntg


%% Bibliographiestil %%%%%%%%%%%%%%%%%%%%%%%%%%%%%%%%%%%%%%%%%%%%%%%%%%
\usepackage{natbib}

\begin{document}
\parskip=1em
\parindent=0cm
\pagestyle{empty} %%Keine Kopf-/Fusszeilen auf den ersten Seiten.


%%%%%%%%%%%%%%%%%%%%%%%%%%%%%%%%%%%%%%%%%%%%%%%%%%%%%%%%%%%%%%%%%%%%%%%
%% Ihr Artikel                                                       %%
%%%%%%%%%%%%%%%%%%%%%%%%%%%%%%%%%%%%%%%%%%%%%%%%%%%%%%%%%%%%%%%%%%%%%%%

%% eigene Titelseitengestaltung %%%%%%%%%%%%%%%%%%%%%%%%%%%%%%%%%%%%%%%    
%\begin{titlepage}
%Einsetzen der TXC Vorlage "Deckblatt" möglich
%description: Deckblatt in Deutsch
%% Basierend auf einer TeXnicCenter-Vorlage von Tino Weinkauf.
%%%%%%%%%%%%%%%%%%%%%%%%%%%%%%%%%%%%%%%%%%%%%%%%%%%%%%%%%%%%%%

%%%%%%%%%%%%%%%%%%%%%%%%%%%%%%%%%%%%%%%%%%%%%%%%%%%%%%%%%%%%%
%% Deckblatt
%%%%%%%%%%%%%%%%%%%%%%%%%%%%%%%%%%%%%%%%%%%%%%%%%%%%%%%%%%%%%
%%
%% ACHTUNG: Sie ben�tigen ein Hauptdokument, um diese Datei
%%          benutzen zu k�nnen. Verwenden Sie im Hauptdokument
%%          den Befehl "\input{dateiname}", um diese
%%          Datei einzubinden.
%%

\begin{titlepage}
\vspace{5cm}
Stefan Waidele\\
Immatrikulationsnummer 1028171\\
Ensisheimer Stra�e 2\\
79395 Neuenburg am Rhein\\
Stefan.Waidele@AKAD.de

\vfill

\begin{tabbing}
Modul ANS09 --- \=Betriebswirtschaftliche Standardsoftware\\
                \>und Enterprise Application Integration (EAI)\\ \\
                \>Assignment  
\end{tabbing}
\LARGE
\textsc{Integrationsanforderungen von \\Informationssystemen am Beispiel\\eines Hotel-- und Restaurantbetriebs}\\

\vfill

\normalsize
Betreuer: Prof. Ulrich Gr�ff

\today %%Datum der Abgabe - am besten selbst reinschreiben.

\vfill

\includegraphics[width=3cm]{AKAD-Logo}

AKAD Hochschule Stuttgart

\end{titlepage}
%\end{titlepage}

%% Erzeugung von Verzeichnissen %%%%%%%%%%%%%%%%%%%%%%%%%%%%%%%%%%%%%%%
\tableofcontents			% Inhaltsverzeichnis
%\listoftables				% Tabellenverzeichnis
\listoffigures				% Abbildungsverzeichnis
\newpage


%% Der Text %%%%%%%%%%%%%%%%%%%%%%%%%%%%%%%%%%%%%%%%%%%%%%%%%%%%%%%%%%%
\pagestyle{headings}
\section{Einleitung}
\subsection{Ziel der Untersuchung}

Ziel dieser Arbeit ist die Analyse und Bewertung der Integrationsanforderungen des Hotelinformationssystems anhand definierter Anforderungsstufen und Integrationsgegenst�nden.

\subsection{Vorgehensweise}

Eingangs des 2. Kapitels erfolgt eine kurze Beschreibung der f�r diese Arbeit relevanten Eigenschaften des Unternehmens. Danach werden die f�r das Verst�ndnis der vorliegenden Arbeit wichtigen Begriffe erl�utert. Mit der Herausarbeitung von Anforderungsstufen und der Integrationsgegenst�nde schlie�t der theoretische Teil ab.

Zu Beginn des praktischen Teils werden die zwei zu betrachtenden Gesch�ftsprozesse beschrieben. Danach folgen die wichtigsten Schnittstellen bzw. Medienbr�che. Den Abschluss dieses Kapitels stellt eine Bewertung  der Schnittstellen anhand der im 2. Kapitel beschriebenen Kriterien. Zum Abschluss der Arbeit werden die Ergebnisse zusammengefasst.

\subsection{Abgrenzung}

Nicht alle unterst�tzte Gesch�ftsprozesse des Hotelinformationssystems flie�en in die Analyse ein. In dieser Arbeit soll das Hauptaugenmerk auf zwei Kernprozesse im Unternehmen gelegt werden, anhand derer die Integrationsanforderungen in einem kleinen Betrieb deutlich werden. Diese werden auch nicht formell modelliert werden. Das Augenmerk der Betrachtungen liegt an den zu erfassenden Daten, deren Weitergabe und die daf�r notwendigen Schnittstellen bzw. die bestehenden Medienbr�chen. 

Aufgrund des sehr geringen Integrationsgrad, welcher f�r Gastronomische Betrieber dieser Gr��enordnung durchaus �blich ist, liegt das Hauptaugenmerk dieser Arbeit auf der Datenintegration. Auch wird keine Wirtschaftlichkeitsberechnung durchgef�hrt. 

Auch wird der fachliche und der Marketingaspekt in dieser Arbeit nicht betrachtet, da diese nur geringen Einfluss auf die Integration haben. Die ebenfalls im Betrieb vorhandenen Abteilungen "`Metzgerei"' und "`Tagungen"' bleiben g�nzlich unbeachtet.

Diese Arbeit besch�ftigt sich mit den folgenden Situationen der betrieblichen Leistungserbringung:

\begin{itemize}
	\item \textbf{Verkauf von Speisen \& Getr�nken im Restaurant:}\\
	Dieser Prozess beschreibt einen gro�en Ausschnitt der kundenzugewandte Seiten der Leistungserstellung im Restaurant. F�r diese Arbeit relevent sind die M�glichkeiten der abteilungs�bergreifenden Integration sowie die Schnittstellen f�r die Daten�bergabe an die Hotelabteilung (horizontale Integration) bzw. an das Management (vertikale Integration) betrachtet.
	\item \textbf{Hotelreservierung}\\
Hier stehen weniger die Erfassung und Verwaltung der G�stedaten und Reservierungen im Blickpunkt, sondern der Umgang mit den Informationen �ber noch zu vermietende Zimmer und die anzusetzenden Preise.
\end{itemize}


\clearpage
\section{Grundlagen}

\subsection{Vorstellung des Unternehmens}

Der Autor untersucht in dieser Arbeit das im eigenen Unternehmen eingesetzte Hotel-- und Restaurantinformationssystem. 
Das Unternehmen ist ein �ber mehere Generationen in Familienbesitz gef�hrtes Einzelunternehmen, 
bestehend aus den Abteilungen Hotel, Restaurant, Metzgerei und Tagungsbereich. Mit 90 Betten und 100 Sitzpl�tzen ist
der Betrieb als "`kleines Hotel"' einzustufen. Mit ca. 30 Mitarbeitern ist das Unternehmen allerdings auch kein reiner Familienbetrieb mehr.

\subsection{Wichtige Fachtermini und Abk�rzungen}

\begin{itemize}
	\item \textbf{POS:} Point of Sale --- Verkaufsstelle. In der Gastronomie wird mit dem Begriff POS--System i.d.R. die Registrierkasse im Restaurant bezeichnet, in der die Bestellung erfasst, gespeichert und auf den verschiedenen Bondrucker ausgegeben wird.
	\item \textbf{Front--Office:} Im Hotel bezeichnet man mit Front--Office den Bereich der Rezeption, an dem der Mitarbeiter in direktem Kontakt zum Kunden steht. F�r diese Arbeit schlie�t dies nicht nur den pers�nlichen Kontakt vor Ort ein, sondern auch die Bereiche der Reservierung und sonstige Korrespondenz, welche in gro�en Hotels auch im \textbf{Back--Office} (also dem B�ro ohne direkten G�stekontakt) oder auch der Abteilung Reservierung zugeordnet sein k�nnen.
	
Trotz deutlichen Unterschieden in der Benutzung und in der Wahrnehmung k�nnen die Begriffe POS-System und Front--Office System somit abteilungsspezifische Bezeichnungen f�r �quivalente Systeme gesehen werden, bei denen der Kundenauftrag f�r die weitere Verarbeitung per EDV erfasst wird. 
	
	\item \textbf{Hotel-- und Restaurantinformationssystem (HRIS):} Betriebliches Informationssystem das die Anforderungen eines Hotels mit Restaurantbetriebs erf�llt.
	\item \textbf{Service Oriented Architecture / SOA:} TODO: Definition!!!
\end{itemize}

\subsection{Anforderungsstufen}

Im Rahmen dieser Arbeit werden die Integrationsanforderungen des Hotel-- und Restaurantinformationssystems auf verschiedenen Stufen betrachtet werden:

\begin{itemize}
	\item Minimalanforderung: Diese beschreiben den Mindestgrad an Integration die f�r einen Betrieb dieser Gr��enordnung wirtschaftlich Sinnvoll ist. 
	\item Momentan umgesetzte Ingegrationsanforderungen: Dies beschreibt den Ist--Zustand im untersuchten Betrieb.
	\item Optimalanforderungen: Hier werden die M�glichkeiten eines hochintegrierten HRIS aufgezeigt.
\end{itemize}

\subsection{Integration}

Integration bedeutet die Verbindung einer Vielheit zu einer Einheit.\footnote{vgl. \cite{dudenint}}

In der Informatik ist demnach Integration als die Verbindung von mehreren getrennten Anwendungssystemen zu einem integrierten Anwendungssystem, welches Aufgaben aus verschiedenen Funktionsbereichen und die verschiedenen Bereiche intern zu einem Gesamtsystem verkn�pft. Hierbei werden Daten m�glichst fr�h erfasst und dann systemintern verarbeitet, gespeichert und weitergeleitet.\footnote{vgl. \cite{staud}, S.30}

Die Integration von Informationssystemen kann auf verschiedenen Arten geschehen. Hierbei sind die unterschiedlichen  Integrationsgegenst�nde jeweils mit den Integrationsrichtungen sowie die der Integrationsreichweite zu betrachten.

\subsection{Integrationsgegenst�nde}

vgl. \cite{staud} Seite 36f

\begin{itemize}
	\item \textbf{Datenintegration:} Hierbei werden die Daten meherer Betriebsbereiche zusammengef�hrt, um dann mit verschiedenen Programmen auf die gemeinsame Datenbasis zugreifen zu k�nnen. \footnote{vgl. \cite{gablerdint}}
	\item \textbf{Funktionsintegration:} Hierbei werden meherere betriebliche Teilfunktionen zusammengef�hrt. So k�nnen vorher getrennte Aufgaben anschlie�end an einem Arbeitsplatz ausgef�hrt werden. \footnote{\cite{gablerfint}}
	\item \textbf{Prozessintegration:} Informationstechnische Verbindung zwischen einzelnen Vorg�ngen, z.B. Auftragserfassung und Materialbeschaffung.
	\item \textbf{Methodenintegration:} Die einzelnen Funktionen werden so gesteltet, dass die Ergebnisse von Funktionen direkt als Eingabe der n�chsten Funktionen genutzt werden k�nnen.
	\item \textbf{Programmintegration:} Realisierung der aufeinander abgestimmten Softwaremodule. Aufgabe des Software--Engineering.
\end{itemize}

\subsection{Integrationsrichtungen}

Bei der horizontalen Integration werden (Teil)prozesse einer Managementebene aus verschiedenen Funktionsbereichen des Unternehmens miteinander verkn�pft. Bei der Vertikalen Integration werden die Systeme �ber die grenzen der Managementebenen hinaus verkn�pft, so dass z.B. das F�hrungsinformationssystem direkt auf die Daten der Produktion zugreigen kann.\footnote{vgl. \cite{staud} Seite 37}

\subsection{Integrationsreichweite}

Hierunter versteht man die l�nge der Integrierten Prozessketten. Dies beginnt von abteilungsinternen Integrationsschritten �ber die Integration von Vorg�ngen �ber mehrere Abteilungen hinweg, bis hin zu Unternehmens�bergreifenden Projekten zum Suppy Chain Management, in denen sowohl Zulieferer als auch Kunden beteiligt sein k�nnen.
%\clearpage

\section{Verkauf von Speisen \& Getr�nken im Restaurant}

Das im Einsatz befindliche POS-System besteht aus einer propriet�ren Restaurantkasse\footnote{Modell "`Casio QT 6600"'; weitere Informationen unter http://www.casio-europe.com/de/ecr/qt6600/} (kurz: POS-Terminal oder Terminal), in der sowohl die Stammdaten als auch die im laufenden Betrieb anfallenden Bewegungsdaten gespeichert werden. Auch das Reporting und eine beschr�nkte Archivierungsfunktion wird erf�llt.

Die Stammdaten basieren auf einer Artikelliste in der neben der Artikelnummer auch eine Beschreibung sowie der Preis gespeichert ist. Aber auch steuerlich relevante Daten wie der Mehrwertsteuersatz sowie Parameter des Customizing (z.B. auf welchem Bondrucker die Bestellung ausgegeben wird) sind hier hinterlegt. 

Nach der Bestellannahme bucht die Servicekraft die gew�nschten Artikel (Auswahl per Men� oder Artikelnummer) auf den Tisch der G�ste. Das System speichert automatisch eine eindeutige Bonnummer, den zum Erfassungszeitpunkt g�ltigen Preis und die Personalnummer der Servicekraft. Diese Daten werden auch auf den ausgedruckten Bons sowie dem intern gespeichertem Journalspeicher dokumentiert.

Des Weiteren werden die angelegten Berichte weitergef�hrt. Dies ist in der Regel der Umsatzbericht pro Servicekraft und der Warengruppenbericht. Weitere Auswertungen z.B. nach Zeitintervallen k�nnen konfiguriert werden. 

\subsection{Schnittstelle POS-System zur Buchhaltung bzw. zum Management}
\subsubsection{Beschreibung}

Die ausgedruckten Berichte stellen die Schnittstelle zur Buchhaltung und zum Management dar. Zur weiteren Verarbeitung m�ssen die ausgedruckten Daten in die entsprechenden Systeme eingegeben werden! 

Auch ist zu beachten, dass die Berichtsdaten zwar laufend aktualisiert werden, es jedoch notwendig ist, die Berichte manuell auszudrucken. Hierbei sind zwei Modi m�glich: Der X--Abschlag, der die aufgelaufenen Daten auswertet und ausgibt, sowie der Z--Abschlag, der die Register nach dem Ausdruck auf Null setzt. Beiden Modi ist gemeinsam, dass die Auswertung der Daten nur zum Zeitpunkt des Ausdrucks ausgewertet werden. 

So ist ein Tagesbericht tats�chlich nur zwischen dem Ende der letzen Schicht und vor dem Beginn der ersten Schicht des Folgetages m�glich. Der Monatsbericht muss nach dem Tagesabschluss des letzten Tags des Monats, vor dem ersten Gesch�ftsvorgang des Folgemonats durchgef�hrt werden. Wird dies nicht beachtet, so wird die Qualit�t der Daten aufgrund der uneinheitlichen Zeitr�umen deutlich verschlechtert.

\subsubsection{Bewertung}

Da die ausgedruckten Daten manuell in das Buchhaltungssystem \footnote{"`Lexware Kassenbuch"', "`Lexware Faktura+"' und Excel--Tabellen} eingegeben werden m�ssen, liegt hier ein \textbf{Medienbruch} vor. Das gleiche gilt f�r die Daten�bergabe an das Management, wo weitere Auswertungen, Planungen und Kalkulationen im Tabellenverarbeitungssystem erstellt werden.

Dieser Medienbruch wird auch dadurch nicht behoben, dass im Back--Office eine Software \footnote{"`Casio Easy Reporter"', kurz: CEP} im Einsatz ist, die auf die entsprechenden Daten der Kasse zugreifen kann. Dieser Zugriff geschieht n�mlich auch nur manuell und in unstrukturierter Form, so dass auch hier umfangreiche Importma�nahmen notwendig sind und auch weiterhin die Notwendigkeit besteht, die Auswertungen zu den entsprechenden Zeitpunkten p�nktlich durchzuf�hren.

Eine integrierte Datenspeicherung w�rde hier einerseits eine Erleichterung der t�glichen Verwaltungsaufgaben mit sich bringen und andererseits auch die Qualit�t der Berichte sicherstellen, da diese nicht mehr an den Perioden�berg�ngen erstellt werden m�ssen. 

\subsection{Schnittstelle Management zum POS--System}
\label{intposman}
\subsubsection{Beschreibung}

Sowohl das Angebot an Speisen und Getr�nken als auch die Preise werden vom Management bestimmt. Grundlage heirf�r ist die Kalkukulation, die im Tabellenverarbeitungssystem durchgef�hrt wird. Die ermittelten Preise werden dann in der Schnittstellenanwendung dem jeweiligen Artikel zugeordnet. Hier werden auch Einstellungen bez�glich der Mehrwertsteuer, eventuellen Optionen (z.B. "`medium"' oder "`durch"') und den zu benutzten Bondruckern get�tigt.

Das POS--System wird allerdings auch ohne die Schnittstellensoftware verkauft. Daher k�nnen die gleichen Einstellungen auch direkt am POS--Terminal get�tigt werden. Von dieser M�glichkeit wird Gebrauch gemacht, wenn einzelne Preise schnell ge�ndert werden sollen\footnote{z.B. bei der t�glichen Eingabe der Preise vom Chefkoch bestimmten Preise f�r das Tagesmen� durch die Servierkr�fte}.

Obwohl das POS-Terminal und die Schnittstellenanwendung vom gleichen Hersteller stammen und aufeinander abgestimmt sind, werden die jeweiligen Daten unabh�ngig voneinander gespeichert. Eine �bertragung vom bzw. zum Terminal findet nur nach einem entsprechenden Befehl statt. Der Datenabgleich muss somit immer manuell angesto�en werden. Eventuelle Synchronisierungskonflikte werden hierbei nicht angezeigt. Hierdurch k�nnen aktualisierte Daten unbemerkt wieder durch alte Versionen �berschrieben werden.

\subsubsection{Bewertung}

Da hier deutlich weniger bzw. seltener Daten �bertragen werden ist der \textbf{Medienbruch} zwischen Management und Restaurantkasse weniger gravierend als der im vorhergehenden Abschnitt beschriebene zur Buchhaltung.
Jedoch bringt die \textbf{redundante Datenhaltung} weitreichende und schwer zu erkennende Synchronisationsprobleme mit sich. 

Eine integrierte L�sung w�rde hier sowohl eine Arbeitserleichterung als auch Schutz vor Datenverlust bringen.

\subsection{Schnittstelle POS--System zur Hotelsoftware}
\label{intposhot}
\subsubsection{Beschreibung}

Wenn ein Hotelgast seine Restaurantrechnung zusammen mit seinem Zimmer bezahlen m�chte, dann muss diese sowohl im POS--Terminal als auch im Hotelprogramm entsprechend verbucht werden. Hierbei flie�en Daten in beide Richtungen: Zun�chst meldet das Hotelprogramm alle belegten Zimmer mit den Gastnamen an die Restaurantkasse. Diese stehen dann dem Servicepersonal in einer Auswahlliste zur Verf�gung. Nach der Auswahl der Zimmernummer wird der Rechnungsbetrag, der Zeitstempel sowie die Rechnungsnummer an das Front--Office--System �bermittelt, welches die entsprechende Buchung auf der Zimmerrechnung vornimmt.

Hierzu arbeiten zwei Programmodule zusammen: Eines ist Bestandteil der Schnittstellensoftware des POS--Systems, das andere ist Bestandteil der Hotelsoftware. Die Daten werden in ein festgelegtes Format gewandelt und �ber zwei Dateien ausgetauscht. Zeitgleich wird der Beleg ausgedruckt und dem Gast zur Pr�fung und Unterschrift vorgelegt.

Eine weitere Kommunikation findet nicht statt. Wird der so abgerechnete Tisch im POS--Terminal wieder aktiviert (i.d.R. ist das notwendig, wenn Fehler korrigiert werden m�ssen), dann wird die Verbuchung im Hotelsystem nicht automatisch r�ckg�ngig gemacht. Die Korrektur muss also von der Servicekraft an den Rezeptionsmitarbeiter gemeldet werden, damit diese im Hotelprogramm manuell nachvollzogen werden kann.

\subsubsection{Bewertung}

Hier kommt im Vergleich zur ebenfalls denkbaren manuellen Daten�bernahme zwischen den Systemen bereits ein \textbf{integrierter Prozess} zum Einsatz, die in den meisten F�llen eine korrekte und konsistente Rechnungsstellung und Buchhaltung gew�rleistet. Lediglich in Sonderf�llen ist ein manuelles Eingreifen m�glich.


%\clearpage
\section{Hotelreservierung}

Das im Einsatz befindliche Property Management System\footnote{"`HS/3 Hotelsoftware"'; weitere Informationen unter http://www.hs3-hotelsoftware.de/} vereint verschiedene Funktionen der Module "`Rezeption"', "`Finanzen"', "`Marketing"' und "`Reporting"' unter einer einheitlichen Programmoberfl�che. Die Module nutzen auch eine einheitliche Datenbasis in Form einer SQL--Datenbank, welche durch ein nur dem Hersteller bekannten Passworts gegen Zugriffe von au�en gesch�tzt ist.

Es handelt sich hierbei also um ein System, in dem die Prinzipien der Daten--, Funktions-- und Programmintegration ber�cksichtigt werden. Dies bildet auch den Alltag des beschriebenen Betriebs ab, da in dieser Betriebsgr��e die abgedeckten Funktionen nicht von getrennten Abteilungen, sondern von einem oder wenigen abteilungs�bergreifend eingesetzten Mitarbeitern ausgef�hrt werden.

F�r einen reinen Hotelbetrieb ohne weitere Betriebszweige ist dieses System auch vom Funktionsumfang her ausreichend. Es k�nnen die Aufgaben der Buchhaltung in Form eines Kassenbuchs und einer Debitorenverwaltung erf�llt werden. Au�erdem wird das Management durch eine Reihe von Auswertungen und Berichten unterst�tzt. Integrationsbedarf entsteht also haupts�chlich dadurch, dass der Betrieb durch seine Gr��e mehrere Betriebszweige vereint.

G�ste haben die M�glichkeit, Zimmer direkt beim Hotel, �ber ein Buchungsformular auf der hoteleigenen Website oder �ber die Buchungsportale "`HRS.de"' bzw. "`Booking.com"' zu reservieren.

\subsection{Schnittstelle zur Buchhaltung}
\subsubsection{Beschreibung}
Da die Buchhaltungsdaten des Restaurants und der anderen Betriebszweige nicht in das PMS eingegeben werden k�nnen, m�ssen die Daten des Hotelprogramms an die Buchhaltung �bergeben werden. Dies geschieht auch hier in Form von ausgedruckten Tages-- und Monatsberichten, die ins Buchhaltungssystem eingetragen werden. 

\subsubsection{Bewertung}
Hier liegt ein \textbf{Medienbruch} vor. Die Buchhaltungssoftware unterst�tzt zwar den Import der Daten im DATEV-Format. Die Hotelsoftware bietet jedoch keinen entsprechenden Export an.

Somit treten hier die Probleme der manuellen Daten�bernahme zu Tage: Arbeitsaufwand, Fehleranf�lligkeit und Inkonsistenzen wegen doppelter Datenhaltung.

\subsection{Schnittstelle zum Management}
\subsubsection{Beschreibung}
Aufgrund der integrierten Informations- und Berichtsfunktionen ist hier zun�chst keine Schnittstelle notwendig. Falls die mit dem Programm ausgelieferten Berichte nicht ausreichen, k�nnen vom Hersteller weitere, individuell programmierte Berichte eingebunden werden. Im Gegensatz zum Restaurant lassen sich diese Berichte jederzeit f�r beliebige Perioden generieren.

Sollen die so aus den Bewegungsdaten gewonnenen Auswertungen weiter verarbeitet werden, ist der Export in eine Tabellenkalkulation m�glich.  

\subsubsection{Bewertung}

Aufgrund des in diesem Bereich \textbf{hohen Integrationsgrads} sind f�r die Information des Hotelmanagement keine weiteren Schnittstellen oder Integrationsschritte notwendig.  

Sobald jedoch die Hotelinformationen mit den Daten der anderen Betriebszweigen verkn�pft werden sollen, so besteht auch hier ein Medienbruch und es ist eine manuelle Daten�bernahme notwendig. Dies gilt auch f�r den Einsatz von EDV--gest�tzen Managementsystemen z.B. f�r das Revenue-- oder Yield--Management zur Optimierung der Kombination Preis--Auslastung des Hotels\footnote{vgl. \cite[8]{goerlich}}.

\subsection{Schnittstelle zu Buchungsportalen im Internet}

Das Internet hat sich zu einem bedeutenden Vertriebsweg f�r Hotelbuchungen entwickelt. Hierbei ist zu beachten, dass es sich hierbei nicht um einen homogenen Vertriebskanal handelt, sondern verschiedene Vertriebskan�le nutzen das Internet als Medium. Aufgrund der Anzahl von beteiligten Gesch�ftspartnern ist der m�gliche Nutzen eines hohen Integrationsniveaus besonders gro�.

\subsubsection{Beschreibung}

Hotelg�ste buchen im Internet i.d.R. �ber das Web--Interface einer Online--Buchungs\-plattform. Hierf�r ist es notwendig, dass das Hotel seine verf�gbaren Zimmer und die entsprechenden Preise (Raten) f�r jede buchbare Nacht an die Buchungsplattform meldet. Die get�tigten Buchungen werden dann an das Hotel weitergegeben und dort im Front--Office--System eingebucht. Hierbei ist beachtenswert, dass das Buchungsportal die Zimmerreservierung dem Gast gegen�ber alleine aufgrund der gemeldeten Verf�gbarkeit best�tigt. Bei manueller Meldung besteht hier bereits bei einer Buchungsplattform die Gefahr, dass es aufgrund von Verz�gerungen im Ablauf zu �berbuchungen oder entgangenen Ums�tzen kommen kann.

Im betrachteten Betrieb werden zwei Buchungsplattformen\footnote{"`HRS.de"' und "`Booking.com"'} genutzt. Des weiteren ist das Hotel �ber ein Buchungsformular auf der eigenen Website\footnote{Buchungsformular unter http://krone-neuenburg.de/hotel/reservation/} buchbar. Somit m�ssen drei Systeme mit m�glichst aktuellen Daten des Front--Officesystems versorgt werden.

Diese Aufgabe wird von einem weiteren System, dem Channel--Manager �bernommen. Das Hotelprogramm sendet Raten und Verf�gbarkeiten nur an bzw. empf�ngt Buchungen nur von diesem einen System. Der Channel--Manager\footnote{"`CultSwich"'; weitere Informationen unter http://www.cultuzz.de/} ist somit die Schnittstelle zu den verschiedenen Buchungssystemen. Er stellt auch die Buchungsmaske zur Verf�gung, die auf der hoteleigenen Website eingebunden werden kann. Die Buchungen der G�ste werden �ber den Channel--Manager direkt in das Front--Officesystem eingebucht. Dieser Datenfluss ist in Abbildung \ref{figih} skizziert. 

\begin{figure}[h!t]
\includegraphics[width=\hsize]{IH}
\caption[Informationsfl�sse bei Hotelbuchungen]{Informationsfl�sse bei Hotelbuchungen, basierend auf \cite{cultuzz}}
\label{figih}
\end{figure}

Die meisten der genutzten Systeme nutzen zum Datenaustausch das vom Branchenverband OpenTravel definierte OTA XML--Schema\footnote{vgl. \cite{otafaq}}. Der Vollst�ndigkeit wegen enth�lt diese Zeichnung noch die Direktbuchung, bei der der Gast pers�nlich, per Telefon oder auch schriftlich (Brief, Fax oder E--Mail) mit dem Hotel in Kontakt tritt. Als weiterer Buchungsweg ist auch die Anbindung von Reiseb�ros �ber eine propriet�re Schnittstelle\footnote{vgl. \cite{amadeusir}} dargestellt.


\subsubsection{Bewertung}

Wie in Abbildung \ref{figih} zu sehen ist, stehen zwischen Hotel und Gast je nach Buchungsweg bis zu drei Mittler. Jedes dieser Systeme wird von seperaten Unternehmen betrieben, womit sich, Hotel und Gast eingerechnet, eine Buchungskette mit f�nf beteiligten Parteien und deren Interessen besteht. Dieser Prozess l�uft hochintegriert ab. Die Daten werden fr�h (direkt vom Gast) erfasst und anschlie�end ohne weitere Benutzereingriffe bis hin zum PMS weitergeleitet.

Bei dieser \textbf{Prozessintegration �ber mehrere Unternehmensgrenzen} hinweg spielt der Channel--Manager eine zentrale Rolle: Zun�chst ist er diejenige Komponente des Systems, die f�r den t�glichen Arbeitsablauf im Hotel den gr��ten Nutzen bringt. Aufgrund der Tatsache, dass Buchungsplattformen keine direkte Schnittstelle zum Hotelprogrammen anbieten, wird durch den Channel--Manager die Automatisierung des Datenabgleichs �berhaupt erst m�glich. Desweiteren wird die Anzahl der notwendigen Schnittstellen von einer pro Buchungskanal auf eine einzige f�r den Channel--Manager reduziert.

Die Tatsache, dass der Datenaustausch bei den eingesetzten Systemen mit dem XML--Branchenstandard OTA von statten geht, ist f�r den Betrieb zun�chst nicht von direktem Interesse. Indirekt werden durch das einheitliche XML--Schema alle beteiligten Komponenten, auch der Channel--Manager selbst, austauschbar. F�r den Channel--Manager ist es ohne Probleme m�glich, weitere Buchungskan�le auf Basis dieses Schemas zu unterst�tzen. F�r Buchungskan�le, die ein anderes Format f�r den Datenaustausch verwenden, muss eine Schnittstelle erstellt werden.

Das Einbinden weiterer vom Channel--Manager unterst�tzter Buchungskan�le ist technisch somit kein Problem. Es ist  im wesentlichen eine Entscheidung der Hotelleitung, die zwischen den zu erwartenden Buchungen und der Provision\footnote{Buchungsportale berechnen i.d.R. 15\% des gebuchten Umsatzes, der Channel--Manager nochmals 0,8\% (Stand Dezember 2012)} des Anbieters abw�gen muss.

Betrachtet man den Markt aus Abbildung 1 als ein Gesamtsystem, dann nimmt der Channel--Manager die Rolle einer \textbf{Middleware} ein, die nach dem Hub--and--Spokes Prinzip zwischen verschiedenen Komponenten --- n�mlich den PMS verschiedener Hotels einerseits und den verschiedenen Buchungsm�glichkeiten andererseits --- vermittelt.

Jedes der beteiligten Systeme h�lt die ben�tigten Daten jedoch in eigenen Datenbanken vor. Aufgrund der \textbf{fehlenden Datenintegration} ist es durch �bermittlungsfehler oder mehreren, gleichzeitigen Buchungen �ber unterschiedliche Portale m�glich, dass Zimmer best�tigt aber nicht gebucht werden bzw. dass es zu �berbuchungen kommt.


%\clearpage
\section{Weitergehende Integrationsma�nahmen}

Die Integration des Hotel-- und Restaurant Informationssystems kann durch folgende Schritte verbessert werden. Die notwendigen �nderungen sind hierzu jedoch nicht vom Betrieb selbst auszuf�hren, sondern bed�rfen �nderungen an der verwendeten Software, die nur vom entsprechenden Hersteller oder im Fall von Zusatzprogrammen von Dritten geleistet werden kann.

\subsection{Datanaustausch �ber gemeinsame Dateiformate}

Die meisten der eingesetzten Programme verf�gen �ber M�glichkeites des Datenim-- und exports. Jedoch nur in den wenigsten F�llen werden hierbei passende Formate unterst�tzt. So w�rde ein Export der Umsatzdaten aus den Hotel bzw. Restaurantsystemen im DATEV-- oder CSV--Format die �bergabe an die Buchhaltung deutlich erleichtern. Pro Monat w�rde hierdurch alleine die manuelle Eingabe von 60 Umsatzwerten sowie die dazugeh�rige �berpr�fung und Fehlersuche entfallen.

Alternativ k�nnte auch ein Konvertierungsprogramm, das zwischen den Exportformaten und dem Importformat erstellt werden.

\subsection{Gemeinsame Datenbasis}

Wie in Abschnitt \ref{intposman} beschrieben, w�rde eine gemeinsame Datenbasis der Restaurantkasse sowie deren Verwaltungsoftware mehr Komfort bei der Eingabe von Artikeln und Preisen mit sich bringen. Eine gemeinsame Datenbank von Restaurantkasse und Hotelsoftware w�rde die auf das Zimmer gebuchten Restaurantrechnungen auch bei nachtr�glichen Korrekturen konsistent halten (vgl. Abschnitt \ref{intposhot}).

\subsection{Integration �ber Softwaremodule (SOA)}

Eine genauere Betrachtung der verschiedenen zu speichernden Daten legt eine Realisierung der Programmfunktionen in Modulen nahe. Diese k�nnen dann innerhalb einer serviceorientierten Systemumgebung f�r die entsprechenden Arbeitspl�tze zusammengef�gt werden.

Folgende Aufgabenbereiche sind sowohl im Restaurant als auch im Hotel zu finden:

\begin{itemize}
	\item G�steverwaltung
	\item Reservierungen
	\item Artikelverwaltung
	\item Leistungserfassung
	\item Rechnungsstellung inkl. Debitorenverwaltung
	\item Kassenabrechnung mit Mitarbeitern nach Schichtende
	\item Daten�bergabe an die Finanzbuchhaltung bzw. Integration der Buchhaltungsfunktionen
	\item ...
\end{itemize}

Die jeweiligen Benutzeroberfl�chen m�ssten diese Module evt. abteilungsspezifisch darstellen, aber prinzipiell sind sowohl Hotelzimmer als auch Restauranttisch tempor�r von G�sten belegte Resourcen, denen die in Anspruch genommenen Leistungen und Zahlungen zugeordnet werden.

%\clearpage
\section{Fazit}

Zusammenfassend l�sst sich sagen, dass es sich bei dem im Restaurant eingesetzte System um eine Insell�sung handelt. Lediglich in Einzelaspekten gibt es eine automatische Daten�bergabe.

Im Hotelbereich sind dagegen schon L�sungen auf einer recht hohen Integrationsstufe im Einsatz. Lediglich bei den Schnittstellen zum Restaurant und zur abteilungs�bergreifenden Buchhaltung besteht noch Integrationspotential. 

Durch jeden weiteren Schritt zu einem h�heren Integrationsniveau werden die in dieser Arbeit beschriebenen M�ngel des momentanen Systems durch die Vorteile integrierter Systeme ersetzt. Die Prozesse werden mit weniger Benutzereingriffen, schneller und mit geringerer Fehleranf�lligkeit ausgef�hrt. 

Somit l�sst sich sagen, dass die generellen Anforderungen an die Integration auch bei kleinen gastronomischen Betrieben vorhanden sind. Jedoch ist die Notwendigkeit, die Dringlichkeit und auch die Rentabilit�t einer Umsetzung deutlich geringer als in gro�en Betrieben.

Im allgemeinen ist auch f�r kleine und mittlere Betriebe ein hochintegriertes Hotel-- und Restaurant Informationssystem anzustreben.

\section{Ausblick}

In weiteren Untersuchungen sollte der finanzielle Aspekt hochintegrierter L�sungen mit in Betracht gezogen werden, da sich hohe Investitionen in die EDV speziell bei kleinen Betrieben evt. nicht amortisieren. 

Ebenfalls sollten Komponenten anderer Softwareh�user als Alternativen zu den bislang Eingesetzten auf deren Integrationsniveau untersucht werden. Hierzu z�hlt auch die Untersuchung der in gro�en Hotels eingesetzten Systeme auf deren Eignung zum Einsatz in kleinen H�usern. Eventuell ist auch schon durch den Tausch einzelner Programme ein System realisierbar, dass deutlich besser zusammenarbeitet als das Momentane.

Betriebs�bergreifend ist die n�here Untersuchung der Channel--Manager interessant. In wie weit nehmen diese die Middleware--Funktion war und wie bedeutsam ist ihre Stellung im Gesamtmarkt. Ist die Hub and Spokes Struktur so vorteilhaft, dass gro�e Channel--Manager ihre zentrale Stellung auch in Marktmacht umsetzen k�nnen? Oder bewirkt eine weitere Standardisierung der Kommunikatinsprotokolle dass die eingenommene �bersetzungs-- und Verteilerfunktion bald �berfl�ssig wird? 


%% Bibliographie unter Verwendung von dinnat %%%%%%%%%%%%%%%%%%%%%%%%%%
%\setbibpreamble{Präambel}		% Text vor dem Verzeichnis
\bibliographystyle{natdin}
%\bibliographystyle{plaindin}
\bibliography{Literatur}	% Sie benötigen eine *.bib-Datei

\end{document}
