%\clearpage
\section{Fazit}

Zusammenfassend l�sst sich sagen, dass es sich bei dem im Restaurant eingesetzte System um eine Insell�sung handelt. Lediglich in Einzelaspekten gibt es eine automatische Daten�bergabe.

Im Hotelbereich sind dagegen schon L�sungen auf einer recht hohen Integrationsstufe im Einsatz. Lediglich bei den Schnittstellen zum Restaurant und zur abteilungs�bergreifenden Buchhaltung besteht noch Integrationspotential. 

Durch jeden weiteren Schritt zu einem h�heren Integrationsniveau werden die in dieser Arbeit beschriebenen M�ngel des momentanen Systems durch die Vorteile integrierter Systeme ersetzt. Die Prozesse werden mit weniger Benutzereingriffen, schneller und mit geringerer Fehleranf�lligkeit ausgef�hrt. 

Somit l�sst sich sagen, dass die generellen Anforderungen an die Integration auch bei kleinen gastronomischen Betrieben vorhanden sind. Jedoch ist die Notwendigkeit, die Dringlichkeit und auch die Rentabilit�t einer Umsetzung deutlich geringer als in gro�en Betrieben.

Im allgemeinen ist auch f�r kleine und mittlere Betriebe ein hochintegriertes Hotel-- und Restaurant Informationssystem anzustreben.

\section{Ausblick}

In weiteren Untersuchungen sollte der finanzielle Aspekt hochintegrierter L�sungen mit in Betracht gezogen werden, da sich hohe Investitionen in die EDV speziell bei kleinen Betrieben evt. nicht amortisieren. 

Ebenfalls sollten Komponenten anderer Softwareh�user als Alternativen zu den bislang Eingesetzten auf deren Integrationsniveau untersucht werden. Hierzu z�hlt auch die Untersuchung der in gro�en Hotels eingesetzten Systeme auf deren Eignung zum Einsatz in kleinen H�usern. Eventuell ist auch schon durch den Tausch einzelner Programme ein System realisierbar, dass deutlich besser zusammenarbeitet als das Momentane.

Betriebs�bergreifend ist die n�here Untersuchung der Channel--Manager interessant. In wie weit nehmen diese die Middleware--Funktion war und wie bedeutsam ist ihre Stellung im Gesamtmarkt. Ist die Hub and Spokes Struktur so vorteilhaft, dass gro�e Channel--Manager ihre zentrale Stellung auch in Marktmacht umsetzen k�nnen? Oder bewirkt eine weitere Standardisierung der Kommunikatinsprotokolle dass die eingenommene �bersetzungs-- und Verteilerfunktion bald �berfl�ssig wird? 

\vfill
