\clearpage
\section{Praktischer Teil}

\subsection{Beschreibung des Anwendungsfalls  / Projektauftrags}

TODO: Beschreibung der zu betrachtenden Gesch�ftsprozessen und der genutzten Informationssystemen.

In dieser Arbeit soll das Hauptaugenmerk auf zwei Kernprozesse im Unternehmen gelegt werden, anhand derer die Integrationsanforderungen in einem kleinen Betrieb deutlich werden. Dies geschieht ohne genaue Modellierung des fachlichen oder des Marketingaspekts, da diese nur bedingt Einfluss auf die Integration haben. Das Augenmerk der Betrachtungen liegt an den zu erfassenden Daten, deren Weitergabe und die daf�r notwendigen Schnittstellen. 

\begin{itemize}
	\item \textbf{Verkauf von Speisen \& Getr�nken im Restaurant:}\\
	Dieser Prozess beschreibt einen gro�en Ausschnitt der kundenzugewandte Seiten der Leistungserstellung im Restaurant. F�r diese Arbeit relevent sind die M�glichkeiten der abteilungs�bergreifenden Integration sowie die Schnittstellen f�r die Daten�bergabe an die Hotelabteilung (horizontale Integration) bzw. an das Management (vertikale Integration) betrachtet.
	\item \textbf{Hotelreservierung}\\
Hier stehen weniger die Erfassung und Verwaltung der G�stedaten und Reservierungen im Blickpunkt, sondern der Umgang mit den Informationen �ber noch zu vermietende Zimmer und die anzusetzenden Preise.
\end{itemize}

\subsection{Verkauf von Speisen \& Getr�nken im Restaurant}

TODO: Hier fehlt die eEPK / bzw. die eEPKs / oder doch nicht?

Das im Einsatz befindliche POS-System "`Casio 6000"' besteht aus einer propriet�ren Restaurantkasse, in der sowohl die Stammdaten als auch die im laufenden Betrieb anfallenden Bewegungsdaten gespeichert werden. Auch das Reporting und eine beschr�nkte Archivierungsfunktion wird erf�llt.

Die Stammdaten basieren auf einer Artikelliste in der neben der Artikelnummer auch eine Beschreibung sowie der Preis gespeichert ist. Aber auch steuerlich relevante Daten wie der Mehrwertsteuersatz sowie Parameter des Customizing (z.B. auf welchem Bondrucker die Bestellung ausgegeben wird) sind hier hinterlegt. 

Nach der Bestellannahme bucht die Servicekraft die gew�nschten Artikel (Auswahl per Men� oder Artikelnummer) auf den Tisch der G�ste. Das System speichert automatisch eine eindeutige Bonnummer, den zum Erfassungszeitpunkt g�ltigen Preis und die Personalnummer der Servicekraft. Diese Daten werden auch auf den ausgedruckten Bons sowie dem intern gespeichertem Journalspeicher dokumentiert.

\subsubsection{Schnittstelle zum Management --- Medienbruch}

Des Weiteren werden die angelegten Berichte weitergef�hrt. Dies ist in der Regel der Umsatzbericht pro Servicekraft und der Warengruppenbericht. Weitere Auswertungen z.B. nach Zeitintervallen k�nnen konfiguriert werden. 

Diese Berichte sind die Schnittstelle zur Buchhaltung und zum Management. Zur weiteren Verarbeitung m�ssen die ausgedruckten Daten in die entsprechenden Systeme eingegeben werden! 

Jedoch ist zu beachten, dass die Berichtsdaten zwar laufend aktualisiert werden, es jedoch notwendig ist, die Berichte manuell auszudrucken. Hierbei sind zwei Modi m�glich: Der X--Abschlag, der die aufgelaufenen Daten auswertet und ausgibt, sowie der Z--Abschlag, der die Register nach dem Ausdruck auf Null setzt. Beiden Modi ist gemeinsam, dass die Auswertung der Daten nur zum Zeitpunkt des Ausdrucks ausgewertet werden. 

So ist ein Tagesbericht tats�chlich nur tats�chlich zwischen dem Ende der letzen Schicht und vor dem Beginn der ersten Schicht des Folgetages m�glich. Der Monatsbericht muss nach dem Tagesabschluss des letzten Tags des Monats, vor dem ersten Gesch�ftsvorgang des Folgemonats durchzuf�hren. Wir dies nicht beachtet, so wird die Qualit�t der Daten deutlich verschlechtert.

Dieser Medienbruch wird auch dadurch nicht behoben, dass im Back--Office eine Software im Einsatz ist, die auf die entsprechenden Daten der Kasse zugreifen kann. Dieser Zugriff geschieht n�mlich auch nur manuell und in unstrukturierter Form, so dass auch hier umfangreiche Importma�nahmen notwendig sind und auch weiterhin die Notwendigkeit besteht, die Auswertungen zu den entsprechenden Zeitpunkten p�nktlich durchzuf�hren.

\subsubsection{Schnittstelle zur Hotelsoftware --- Medienbruch}


\subsubsection{Hotelreservierung}

\subsection{Die wichtigsten Schnittstellen}

Herausarbeitung der wichtigsten Schnittstellen zwischen den Systemen und des Nutzens, den das Unternehmen durch Integration erh�lt -> Integrationsanforderungen. 

\subsection{Bewertung der Schnittstellen}

Bewertung der Schnittstellen anhand der Anforderungsstufen und Integrationsgegenst�nden.

