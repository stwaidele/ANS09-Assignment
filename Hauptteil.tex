\clearpage
\section{ Praktischer Teil}

\subsection{Beschreibung des Anwendungsfalls  / Projektauftrags}

TODO: Beschreibung der zu betrachtenden Gesch�ftsprozessen und der genutzten Informationssystemen.

In dieser Arbeit soll das Hauptaugenmerk auf zwei Kernprozesse im Unternehmen gelegt werden, anhand derer die Integrationsanforderungen in einem kleinen Betrieb deutlich werden. Dies geschieht ohne genaue Modellierung des fachlichen oder des Marketingaspekts, da diese nur bedingt Einfluss auf die Integration haben. Das Augenmerk der Betrachtungen liegt an den zu erfassenden Daten, deren Weitergabe und die daf�r notwendigen Schnittstellen. 

\begin{itemize}
	\item \textbf{Verkauf von Speisen \& Getr�nken im Restaurant mit dem POS-Kassensystem:}\\
	Dieser Prozess beschreibt einen gro�en Ausschnitt der kundenzugewandte Seiten der Leistungserstellung im Restaurant. F�r diese Arbeit relevent sind die M�glichkeiten der abteilungs�bergreifenden Integration sowie die Schnittstellen f�r die Daten�bergabe an die Hotelabteilung (horizontale Integration) bzw. an das Management (vertikale Integration) betrachtet.
	\item \textbf{Hotelreservierung}\\
Hier stehen weniger die Erfassung und Verwaltung der G�stedaten und Reservierungen im Blickpunkt, sondern der Umgang mit den Informationen �ber noch zu vermietende Zimmer und die anzusetzenden Preise.
\end{itemize}

\subsection{Die wichtigsten Schnittstellen}

Herausarbeitung der wichtigsten Schnittstellen zwischen den Systemen und des Nutzens, den das Unternehmen durch Integration erh�lt -> Integrationsanforderungen. 

\subsection{Bewertung der Schnittstellen}

Bewertung der Schnittstellen anhand der Anforderungsstufen und Integrationsgegenst�nden.

