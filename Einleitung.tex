\section{Einleitung}
\subsection{Ziel der Untersuchung}

Ziel dieser Arbeit ist die Analyse und Bewertung der Integrationsanforderungen des Hotelinformationssystems anhand definierter Anforderungsstufen und Integrationsgegenst�nden.

\subsection{Vorgehensweise}

Eingangs des 2. Kapitels erfolgt eine kurze Beschreibung der f�r diese Arbeit relevanten Eigenschaften des Unternehmens. Danach werden die f�r das Verst�ndnis der vorliegenden Arbeit wichtigen Begriffe erl�utert. Mit der Herausarbeitung von Anforderungsstufen und der Integrationsgegenst�nde schlie�t der theoretische Teil ab.

Zu Beginn des praktischen Teils werden die zwei zu betrachtenden Gesch�ftsprozesse beschrieben. Danach folgen die wichtigsten Schnittstellen bzw. Medienbr�che. Den Abschluss dieses Kapitels stellt eine Bewertung  der Schnittstellen anhand der im 2. Kapitel beschriebenen Kriterien. Zum Abschluss der Arbeit werden die Ergebnisse zusammengefasst.

\subsection{Abgrenzung}

Nicht alle unterst�tzte Gesch�ftsprozesse des Hotelinformationssystems flie�en in die Analyse ein. In dieser Arbeit soll das Hauptaugenmerk auf zwei Kernprozesse im Unternehmen gelegt werden, anhand derer die Integrationsanforderungen in einem kleinen Betrieb deutlich werden. Diese werden auch nicht formell modelliert werden. Das Augenmerk der Betrachtungen liegt an den zu erfassenden Daten, deren Weitergabe und die daf�r notwendigen Schnittstellen bzw. die bestehenden Medienbr�chen. 

Aufgrund des sehr geringen Integrationsgrad, welcher f�r Gastronomische Betrieber dieser Gr��enordnung durchaus �blich ist, liegt das Hauptaugenmerk dieser Arbeit auf der Datenintegration. Auch wird keine Wirtschaftlichkeitsberechnung durchgef�hrt. 

Auch wird der fachliche und der Marketingaspekt in dieser Arbeit nicht betrachtet, da diese nur geringen Einfluss auf die Integration haben. Die ebenfalls im Betrieb vorhandenen Abteilungen "`Metzgerei"' und "`Tagungen"' bleiben g�nzlich unbeachtet.

Diese Arbeit besch�ftigt sich mit den folgenden Situationen der betrieblichen Leistungserbringung:

\begin{itemize}
	\item \textbf{Verkauf von Speisen \& Getr�nken im Restaurant:}\\
	Dieser Prozess beschreibt einen gro�en Ausschnitt der kundenzugewandte Seiten der Leistungserstellung im Restaurant. F�r diese Arbeit relevent sind die M�glichkeiten der abteilungs�bergreifenden Integration sowie die Schnittstellen f�r die Daten�bergabe an die Hotelabteilung (horizontale Integration) bzw. an das Management (vertikale Integration) betrachtet.
	\item \textbf{Hotelreservierung}\\
Hier stehen weniger die Erfassung und Verwaltung der G�stedaten und Reservierungen im Blickpunkt, sondern der Umgang mit den Informationen �ber noch zu vermietende Zimmer und die anzusetzenden Preise.
\end{itemize}

