\section{Einleitung}
\subsection{Ziel der Untersuchung}

Ziel dieser Arbeit ist die Analyse und Bewertung der Integrationsanforderungen des Hotelinformationssystems anhand definierter Anforderungsstufen und Integrationsgegenst�nden.

\subsection{Vorgehensweise}

Eingangs des 2. Kapitels erfolgt eine kurze Beschreibung der f�r diese Arbeit relevanten Eigenschaften des Unternehmens. Danach werden die f�r das Verst�ndnis der vorliegenden Arbeit wichtigen Begriffe erl�utert. Mit der Herausarbeitung von Anforderungsstufen und der Integrationsgegenst�nde schlie�t der theoretische Teil ab.

Zu Beginn des praktischen Teils werden die zwei zu betrachtenden Gesch�ftsprozesse beschrieben. Danach folgen die wichtigsten Schnittstellen bzw. Medienbr�che. Den Abschluss dieses Kapitels stellt eine Bewertung  der Schnittstellen anhand der im 2. Kapitel beschriebenen Kriterien. Zum Abschluss der Arbeit werden die Ergebnisse zusammengefasst.

\subsection{Abgrenzung}

Nicht alle unterst�tzte Gesch�ftsprozesse des Hotelinformationssystems flie�en in die Analyse ein, sondern nur zwei ausgew�hlte Gesch�ftsprozesse. Diese werden auch nur grob modelliert werden. Lediglich in den f�r die Integration relevanten Bereichen wie den Schnittstellen und Medienbr�chen wird der Detailgrad erh�ht.

Die ebenfalls im Betrieb vorhandenen Abteilungen "`Metzgerei"' und "`Tagungen"' bleiben g�nzlich unbeachtet.