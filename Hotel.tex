\clearpage
\section{Hotelreservierung}

Das im Einsatz befindliche Property Management System (PMS) ist die Hotelsoftware "`HS/3"'. Darin werden verschiedene Funktionen der Module "`Rezeption"', "`Finanzen"', "`Marketing"' und "`Reporting"' unter einer einheitlichen Programmoberfl�che vereint. Die Module nutzen auch eine einheitliche Datenbasis in Form einer SQL--Datenbank, welche durch ein nur dem Hersteller bekannten Passworts gegen Zugriffe von au�en gesch�tzt ist.

Es handelt sich hierbei also um ein System, in dem die Prinzipien der Daten--, Funktions-- und Programmintegration ber�cksichtigt. Dies bildet auch den Alltag des beschriebenen Betriebs ab, da in dieser Betriebsgr��e die abgedeckten Funktionen nicht von getrennten Abteilungen, sondern von einem oder wenigen abteilungs�bergreifend eingesetzten Mitarbeitern ausgef�hrt werden.

F�r einen reinen Hotelbetrieb ohne weitere Betriebszweige ist dieses System auch vom Funktionsumfang her ausreichend. Es k�nnen die Aufgaben der Buchhaltung in Form eines Kassenbuchs und einer Debitorenverwaltung erf�llt werden. Au�erdem wird das Management durch eine Reihe von Auswertungen und Berichten unterst�tzt. Integrationsbedarf entsteht also haupts�chlich dadurch, dass der Betrieb durch seine Gr��e mehrere Betriebszweige vereint.

\subsection{Schnittstelle zur Buchhaltung}
\subsubsection{Beschreibung}
Da die Buchhaltungsdaten des Restaurants und der anderen Betriebszweige nicht in das PMS eingegeben werden k�nnen, m�ssen die Daten des Hotelprogramms an die Buchhaltung �bergeben werden. Dies geschieht auch hier in Form von ausgedruckten Tages-- und Monatsberichten, die ins Buchhaltungssystem eingegeben werden m�ssen. 

\subsubsection{Bewertung}
Hier liegt ein Medienbruch vor. Die Buchhaltungssoftware unterst�tzt zwar den Import der Daten im DATEV-Format. Die Hotelsoftware bietet jedoch keinen entsprechenden Export an.

Somit treten hier die Probleme der manuellen Daten�bernahme zu Tage: Arbeitsaufwand, Fehleranf�lligkeit und Inkonsistenzen wegen doppelter Datenhaltung.

\subsection{Schnittstelle zum Management}
\subsubsection{Beschreibung}
Aufgrund der integrierten Informations- und Berichtsfunktionen ist hier zun�chst keine Schnittstelle notwendig. Falls die mit dem Programm ausgelieferten Berichte nicht ausreichen, k�nnen vom Hersteller weitere, individuell programmierte Berichte eingebunden werden. Im Gegensatz zum Restaurant lassen sich diese Berichte jederzeit f�r beliebige Perioden generieren.

Sollen die so aus den Bewegungsdaten gewonnenen Auswertungen weiter verarbeitet werden, ist der Export in eine Tabellenkalkulation m�glich.  

\subsubsection{Bewertung}

Aufgrund des in diesem Bereich hohen Integrationsgrads sind f�r die Information des Hotelmanagement keine weiteren Schnittstellen oder Integrationsschritte notwendig.  

Sobald jedoch die Hotelinformationen mit den Daten der anderen Betriebszweigen verkn�pft werden sollen, so besteht auch hier ein Medienbruch und es ist eine manuelle Daten�bernahme notwendig. 

\subsection{Schnittstelle zu Buchungsportalen im Internet}
\subsubsection{Beschreibung}



\subsubsection{Bewertung}
Bewertung der Schnittstellen anhand der Anforderungsstufen und Integrationsgegenst�nden.
