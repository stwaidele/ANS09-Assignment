\clearpage
\section{Grundlagen}

\subsection{Vorstellung des Unternehmens}

Der Autor untersucht in dieser Arbeit das im eigenen Unternehmen eingesetzte Hotel-- und Restaurantinformationssystem. 
Das Unternehmen ist ein �ber mehere Generationen in Familienbesitz gef�hrtes Einzelunternehmen, 
bestehend aus den Abteilungen Hotel, Restaurant, Metzgerei und Tagungsbereich. Mit 90 Betten und 100 Sitzpl�tzen ist
der Betrieb als "`kleines Hotel"' einzustufen. Mit ca. 30 Mitarbeitern ist das Unternehmen allerdings auch kein reiner Familienbetrieb mehr.

\subsection{Abk�rzungen und Definitionen}

\begin{itemize}
	\item \textbf{POS:} Point of Sale --- Verkaufsstelle. In der Gastronomie wird mit dem Begriff POS--System i.d.R. die Registrierkasse im Restaurant bezeichnet, in der die Bestellung erfasst, gespeichert und auf den verschiedenen Bondrucker ausgegeben wird.
	\item \textbf{Front--Office:} Im Hotel bezeichnet man mit Front--Office den Bereich der Rezeption, an dem der Mitarbeiter in direktem Kontakt zum Kunden steht. F�r diese Arbeit schlie�t dies nicht nur den pers�nlichen Kontakt vor Ort ein, sondern auch die Bereiche der Reservierung und sonstige Korrespondenz, welche in gro�en Hotels auch im \textbf{Back--Office} (also dem B�ro ohne direkten G�stekontakt) oder auch der Abteilung Reservierung zugeordnet sein k�nnen
	
Trotz deutlichen Unterschieden in der Benutzung und in der Wahrnehmung k�nnen die Begriffe POS-System und Front--Office System somit abteilungsspezifische Bezeichnungen f�r �quivalente Systeme gesehen werden, bei denen der Kundenauftrag f�r die weitere Verarbeitung per EDV erfasst wird. 

\item \textbf{PMS:} Property Management System --- Softwaresystem zur Verwaltung von �bernachtungsm�glichkeiten. Hierbei kann es sich um die wenigen Zimmer einer kleiner Pension bis hin zur einheitlichen Verwaltung der unterschiedlichen Standorte gro�er Hotelketten handeln.
	\item \textbf{Hotel-- und Restaurantinformationssystem (HRIS):} Betriebliches Informationssystem das die Anforderungen eines Hotels mit Restaurantbetriebs erf�llt.
	\item \textbf{Service Oriented Architecture / SOA:} TODO: Definition!!!
\end{itemize}

\subsection{Anforderungsstufen}

Im Rahmen dieser Arbeit werden die Integrationsanforderungen des Hotel-- und Restaurantinformationssystems auf verschiedenen Stufen betrachtet werden:

\begin{itemize}
	\item Minimalanforderung: Diese beschreiben den Mindestgrad an Integration die f�r einen Betrieb dieser Gr��enordnung wirtschaftlich Sinnvoll ist. 
	\item Momentan umgesetzte Ingegrationsanforderungen: Dies beschreibt den Ist--Zustand im untersuchten Betrieb.
	\item Optimalanforderungen: Hier werden die M�glichkeiten eines hochintegrierten HRIS aufgezeigt.
\end{itemize}

\subsection{Integration}

Integration bedeutet die Verbindung einer Vielheit zu einer Einheit.\footnote{vgl. \cite{dudenint}}

In der Informatik ist demnach Integration als die Verbindung von mehreren getrennten Anwendungssystemen zu einem integrierten Anwendungssystem, welches Aufgaben aus verschiedenen Funktionsbereichen und die verschiedenen Bereiche intern zu einem Gesamtsystem verkn�pft. Hierbei werden Daten m�glichst fr�h erfasst und dann systemintern verarbeitet, gespeichert und weitergeleitet.\footnote{vgl. \cite{staud}, S.30}

Die Integration von Informationssystemen kann auf verschiedenen Arten geschehen. Hierbei sind die unterschiedlichen  Integrationsgegenst�nde jeweils mit den Integrationsrichtungen sowie die der Integrationsreichweite zu betrachten.

\subsection{Integrationsgegenst�nde}

vgl. \cite{staud} Seite 36f

\begin{itemize}
	\item \textbf{Datenintegration:} Hierbei werden die Daten meherer Betriebsbereiche zusammengef�hrt, um dann mit verschiedenen Programmen auf die gemeinsame Datenbasis zugreifen zu k�nnen. \footnote{vgl. \cite{gablerdint}}
	\item \textbf{Funktionsintegration:} Hierbei werden meherere betriebliche Teilfunktionen zusammengef�hrt. So k�nnen vorher getrennte Aufgaben anschlie�end an einem Arbeitsplatz ausgef�hrt werden. \footnote{\cite{gablerfint}}
	\item \textbf{Prozessintegration:} Informationstechnische Verbindung zwischen einzelnen Vorg�ngen, z.B. Auftragserfassung und Materialbeschaffung.
	\item \textbf{Methodenintegration:} Die einzelnen Funktionen werden so gesteltet, dass die Ergebnisse von Funktionen direkt als Eingabe der n�chsten Funktionen genutzt werden k�nnen.
	\item \textbf{Programmintegration:} Realisierung der aufeinander abgestimmten Softwaremodule. Aufgabe des Software--Engineering.
\end{itemize}

\subsection{Integrationsrichtungen}

Bei der horizontalen Integration werden (Teil)prozesse einer Managementebene aus verschiedenen Funktionsbereichen des Unternehmens miteinander verkn�pft. Bei der Vertikalen Integration werden die Systeme �ber die grenzen der Managementebenen hinaus verkn�pft, so dass z.B. das F�hrungsinformationssystem direkt auf die Daten der Produktion zugreigen kann.\footnote{vgl. \cite{staud} Seite 37}

\subsection{Integrationsreichweite}

Hierunter versteht man die l�nge der Integrierten Prozessketten. Dies beginnt von abteilungsinternen Integrationsschritten �ber die Integration von Vorg�ngen �ber mehrere Abteilungen hinweg, bis hin zu Unternehmens�bergreifenden Projekten zum Suppy Chain Management, in denen sowohl Zulieferer als auch Kunden beteiligt sein k�nnen.