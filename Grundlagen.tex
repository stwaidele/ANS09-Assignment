\clearpage
\section{Theoretischer / methodischer Teil}

\subsection{Vorstellung des Unternehmens}

Der Autor untersucht in dieser Arbeit das im eigenen Unternehmen eingesetzte Hotel-- und Restaurantinformationssystem. 
Das Unternehmen ist ein �ber mehere Generationen in Familienbesitz gef�hrtes Einzelunternehmen, 
bestehend aus den Abteilungen Hotel, Restaurant, Metzgerei und Tagungsbereich. Mit 90 Betten und 100 Sitzpl�tzen ist
der Betrieb als "`kleines Hotel"' einzustufen. Mit ca. 30 Mitarbeitern ist das Unternehmen allerdings kein reiner Familienbetrieb mehr.

\subsection{Wichtige Fachtermini und Abk�rzungen}

\begin{itemize}
	\item \textbf{Hotel:} Ein Hotel ist ein Beherbergungsbetrieb, in dem eine Rezeption, Dienstleistungen, t�gliche Zimmerreinigung, zus�tzliche Einrichtungen und mind. ein Restaurant f�r Hausg�ste und Passanten angeboten werden. Ein Hotel sollte �ber mehr als 20 G�stezimmer verf�gen.\footnote{\cite{dehogadef}}
	\item \textbf{Hotel-- und Restaurantinformationssystem (HRIS):} Betriebliches Informationssystem das die Anforderungen eines Hotels mit Restaurantbetriebs erf�llt.
\end{itemize}

\subsection{Anforderungsstufen}

Im Rahmen dieser Arbeit werden die Integrationsanforderungen des Hotel-- und Restaurantinformationssystems auf verschiedenen Stufen betrachtet werden:

\begin{itemize}
	\item Minimalanforderung: Diese beschreiben den Mindestgrad an Integration die f�r einen Betrieb dieser Gr��enordnung wirtschaftlich Sinnvoll ist.
	\item Momentan umgesetzte Ingegrationsanforderungen: Dies beschreibt den Ist--Zustand im untersuchten Betrieb.
	\item Optimalanforderungen: Hier werden die M�glichkeiten eines hochintegrierten HRIS aufgezeigt.
\end{itemize}


\subsection{Integrationsgegenst�nde}

Hierbei gehe ich auf die verschiedenen Integrationsgegenst�nde (Daten-, Funktions-, Prozess-, Methoden- und Programmintegration), Integrationsrichtungen (horizontal/vertikal) und die Integrationsreichweite ein.

