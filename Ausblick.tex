%\clearpage
\section{Weitergehende Integrationsma�nahmen}

Die Integration des Hotel-- und Restaurant Informationssystems kann durch folgende Schritte verbessert werden. Diese bed�rfen �nderungen an der verwendeten Software, die nur vom entsprechenden Hersteller oder im Fall von Zusatzprogrammen von Dritten geleistet werden k�nnen.

\subsection{Datanaustausch �ber gemeinsame Dateiformate}

Die meisten der eingesetzten Programme verf�gen �ber M�glichkeites des Datenim-- und exports. Jedoch nur in den wenigsten F�llen werden hierbei passende Formate unterst�tzt. So w�rde ein Export der Umsatzdaten aus den Hotel bzw. Restaurantsystemen im DATEV-- oder CSV--Format die �bergabe an die Buchhaltung deutlich erleichtern. Pro Monat w�rde hierdurch alleine die manuelle Eingabe von 60 Umsatzwerten sowie die dazugeh�rige �berpr�fung und Fehlersuche entfallen.

Alternativ k�nnte auch ein Konvertierungsprogramm, das zwischen den Exportformaten und dem Importformat �bersetzt, erstellt werden.

\subsection{Gemeinsame Datenbasis}

Wie in Abschnitt \ref{intposman} beschrieben, w�rde eine gemeinsame Datenbasis der Restaurantkasse sowie deren Verwaltungsoftware mehr Komfort bei der Eingabe von Artikeln und Preisen mit sich bringen. Eine gemeinsame Datenbank von Restaurantkasse und Hotelsoftware w�rde die auf das Zimmer gebuchten Restaurantrechnungen auch bei nachtr�glichen Korrekturen konsistent halten (vgl. Abschnitt \ref{intposhot}).

\subsection{Integration �ber Softwaremodule (SOA)}

Eine genauere Betrachtung der verschiedenen zu speichernden Daten legt eine Realisierung der Programmfunktionen in Modulen nahe. Diese k�nnen dann innerhalb einer serviceorientierten Systemumgebung f�r die entsprechenden Arbeitspl�tze zusammengef�gt werden.

Folgende Aufgabenbereiche sind sowohl im Restaurant als auch im Hotel zu finden:

\begin{itemize}
	\item G�steverwaltung
	\item Reservierungen
	\item Artikelverwaltung
	\item Leistungserfassung
	\item Rechnungsstellung inkl. Debitorenverwaltung
	\item Kassenabrechnung mit Mitarbeitern nach Schichtende
	\item Daten�bergabe an die Finanzbuchhaltung bzw. Integration der Buchhaltungsfunktionen
	\item ...
\end{itemize}

Die jeweiligen Benutzeroberfl�chen m�ssten diese Module evt. abteilungsspezifisch darstellen, aber prinzipiell sind sowohl Hotelzimmer als auch Restauranttisch tempor�r von G�sten belegte Resourcen, denen die in Anspruch genommenen Leistungen und Zahlungen zugeordnet werden.
